\documentclass[fleqn,reqno,a4paper,parskip=half]{scrartcl}
%\usepackage{showkeys}      % zeigt label-Bezeichner an

%%%%%%%%%%%%%%%%   Pakete   %%%%%%%%%%%%%%%%%%

\usepackage{ifxetex}
\ifxetex                % Pakete für XeLaTex / XeTex
    
    \usepackage{fontspec}
    \defaultfontfeatures{Mapping=tex-text}
    \usepackage{unicode-math}
    
    %------------- Schriftarten: ------------------------------
    %\setmathfont{xits-math.otf}
    %\setmathfont{latinmodern-math.otf}
    %\setmathfont{texgyrepagella-math.otf}
    %\setmathfont{Asana-Math.otf}
    
\else                   % Befehle für pdflatex
%   \usepackage[utf8]{inputenc}


\usepackage[mathletters]{ucs} %direkt griechisches im Mathe modus
\usepackage[utf8x]{inputenc}
%\usepackage[T1]{fontenc}
%\usepackage{times}

%TODO_Lorin: besser?
   %\usepackage{uniinput}       % für Unicode-Zeichen, wird momentan nicht verwendet, deshalb auskommentiert by Benni
\fi


\usepackage[ngerman]{babel}
%\usepackage{ngerman}
\usepackage[tbtags,sumlimits,intlimits,namelimits]{amsmath}

\usepackage{amsfonts}
\usepackage{amssymb}
\usepackage{bbm}
\usepackage{ulem}
\usepackage{tikz}
\usepackage{pgf}
\usepackage{ifpdf}
\usepackage{color}
\usepackage{esint}
\usepackage{framed}
%\usepackage{harmony}   auskommentiert by Georg, da beim compilieren
%                       "harmony.sty not found"
%\usepackage[colorlinks=true,linkcolor=black,citecolor=black,urlcolor=black]{hyperref}  % print
\usepackage[colorlinks=true,linkcolor=blue,citecolor=blue]{hyperref}    % web
\usepackage[top=2.3cm, bottom=3.45cm, left=2.3cm, right=2.3cm]{geometry}
%\numberwithin{equation}{section}
\usepackage{chngcntr}
\counterwithin*{section}{part}
%\graphicspath{{images/png/}{images/}}        % Pfad, in dem sich Grafikdateien befinden
%\usepackage{subfigure}          % Unterbilder, deprecated
%\usepackage(subfig}

\usepackage[all]{hypcap}
\usepackage{cite}           % Literatur
\usepackage{graphicx}       % Bilder in Tabellen
\usepackage{float}          % eigene Float-Umgebungen, H-Option, um Bilder an der aktuellen Stelle anzuzeigen
\usepackage{caption}
\usepackage{subcaption,array}
%\usepackage{subcaption}

\restylefloat{figure}       % Bilder an der Stelle, wo sie eingebunden werden
\usepackage{multirow}
\usepackage{listings}       % Darstellung von Source-Code
\usepackage{framed}         % Rahmen um Text
\usepackage{mdframed}       % Rahmen um Text und Gleichungen
%\usepackage{arydshln}      % gestrichelte Linie in Tabelle mit \hdashline
\usepackage{dirtytalk}          % \say{...} erzeugt (deutsche) Anführungszeichen

\usepackage{tipa}
\usepackage{transparent}    % needed for inkscape generated pdf_tex files
\usepackage{multicol}       % multiple columns
\usepackage{moreverb}       % verbatimwrite
\usepackage{verbatimbox}    % \begin{verbbox}
\usepackage{booktabs}
\usepackage{morefloats}
\usepackage{cleveref}
\usepackage{mathrsfs}       % mathscr

\newsavebox\lstbox
\mdfdefinestyle{MyFrame}{%
    innertopmargin=0pt,
    innerbottommargin=10pt,
    innerrightmargin=20pt,
    innerleftmargin=20pt}

\definecolor{darkgreen}{HTML}{009900}
    
% settings for algorithm
\lstset{literate=%
    {Ö}{{\"O}}1
    {Ä}{{\"A}}1
    {Ü}{{\"U}}1
    {ß}{{\ss}}1
    {ü}{{\"u}}1
    {ä}{{\"a}}1
    {ö}{{\"o}}1
    {⇐}{{$\leftarrow$}}1
    {>=}{{$\geq$}}1
    {~}{{\textasciitilde}}1
    {`}{\textquotedbl}1,  
  language=C++,
  numbers=none,
  numberstyle=\tiny,
  xleftmargin=2.0ex, 
  %basicstyle=\small, %  print  whole  listing  small
  basicstyle=\small\ttfamily,
  morekeywords={elif,do,end,then,proc,local,Eingabe,Ausgabe,alignof,loop,each},
  deletekeywords={new},
  columns=flexible,   % alignment
  tabsize=2,    % size of tabs
  keepspaces,
  gobble=2,    % remove 2 characters at begin of each line
  mathescape    % wandle $$ in latex um
}

% Versuche stärker, Abbildungen dort einzubinden, wo sie definiert wurden
\renewcommand{\topfraction}{.85}      % Anteil, den floats auf einer Seite von oben her einnehmen dürfen
\renewcommand{\bottomfraction}{.7}    % Anteil, den floats auf einer Seite von unten her einnehmen dürfen
\renewcommand{\textfraction}{.15}       % Anteil der Seite, der mind. für Text zur Verfügung steht
\renewcommand{\floatpagefraction}{.66}  % Anteil der Seite, der belegt sein muss, bevor eine weitere Seite angelegt wird
\setcounter{topnumber}{9}               % maximale Anzahl floats, die im oberen Bereich der Seite sein dürfen
\setcounter{bottomnumber}{9}            % maximale Anzahl floats, die im unteren Bereich der Seite sein dürfen
    
\newcommand{\bild}[3]{%
    \begin{figure}%
        \centering%
        \def\svgwidth{#2}%
        \input{images/#1.pdf_tex}%
        \caption{#3}%
        \label{fig:#1}%
    \end{figure}%
}

\newcommand{\subfig}[3]{%
    \begin{subfigure}[b]{#2}%
        \centering%
        \def\svgwidth{#2}%
        \input{images/#1.pdf_tex}%
        \caption{#3}%
        \label{fig:#1}%
    \end{subfigure}%
}

\newcommand{\subfigpng}[3]{%
    \begin{subfigure}[t]{#2}%
        \centering%
        \includegraphics[width=#2]{images/#1.png}%
        \caption{#3}%
        \label{fig:#1}%
    \end{subfigure}%
}
\newcommand{\subfigpngheight}[4]{%
    \begin{subfigure}[t]{#2}%
        \centering%
        \includegraphics[height=#3]{images/#1.png}%
        \caption{#4}%
        \label{fig:#1}%
    \end{subfigure}%
}

\newcommand{\subfigpdf}[3]{%
    \begin{subfigure}[b]{#2}%
        \centering%
        \includegraphics[width=#2]{images/#1.pdf}%
        \caption{#3}%
        \label{fig:#1}%
    \end{subfigure}%
}

\newcommand{\subfigsvg}[3]{%
    \begin{subfigure}[b]{#2}%
        \centering%
        \includegraphics[width=#2]{images/#1.svg}%
        \caption{#3}%
        \label{fig:#1}%
    \end{subfigure}%
}

\newcommand{\bildpng}[3]{%
    \begin{figure}[ht]%
        \centering%
        \includegraphics[width=#2]{images/#1.png}%
        \caption{#3}%
        \label{fig:#1}%
    \end{figure}%
}
\newcommand{\bildsvg}[3]{%
    \begin{figure}[ht]%
        \centering%
        \includegraphics[width=#2]{images/#1.svg}%
        \caption{#3}%
        \label{fig:#1}%
    \end{figure}%
}

\newcommand{\bildpdf}[3]{%
    \begin{figure}%
        \centering%
        \includegraphics[width=#2]{images/#1.pdf}%
        \caption{#3}%
        \label{fig:#1}%
    \end{figure}%
}

%%%%%%%%%%%%%%%%   Abkürzungen   %%%%%%%%%%%%%%%%%%

%----------------------Umgebungen----------------------
\def\beqno{\begin{equation}}
\def\eeqno{\end{equation}}
\def\beq{\begin{equation*}}
\def\eeq{\end{equation*}}
\def\ba#1{\begin{array}{#1}}
\def\ea{\end{array}}
\def\mat#1{\left(\begin{matrix}#1\end{matrix}\right)}   % added by Georg
\def\matt#1{\left[\begin{matrix}#1\end{matrix}\right]}

\newcommand{\Name}[1]   {\textit{#1}\/}                 % Eigennamen kursiv
\renewcommand{\emph}[1]{\textit{#1}\/}
\def\clap#1{\hbox  to  0pt{\hss#1\hss}}                 % für underbrace
\def\mathclap{\mathpalette\mathclapinternal}
\def\mathclapinternal#1#2{\clap{$\mathsurround=0pt#1{#2}$}}
\newcommand{\ub}[2]{\underbrace{#1}_{\mathclap{#2}}}    
\newcommand{\ds}{\displaystyle}                         % displaystyle
\newcommand{\scr}{\scriptstyle}
\renewcommand{\dfrac}[2]{\ds\frac{\ds{#1}}{\ds{#2}}\,}  % nach Bruch Abstand
%\newcommand{\code}[1]{{\small\lstinline[columns=fixed]!#1!}}
\newcommand\undermat[2]{%
\makebox[0pt][l]{$\smash{\underbrace{\phantom{%
\begin{matrix}#2\end{matrix}}}_{\text{$#1$}}}$}#2}


\usepackage{setspace}
\newcommand{\code}[1]{{\small\lstinline[basicstyle=\small\ttfamily\color{Maroon},breaklines=true,upquote=true]!#1!}}
\newcommand{\codenobreak}[1]{{\small\lstinline[basicstyle=\small\ttfamily\color{Maroon},breaklines=false,upquote=true]!#1!}}
\newcommand{\emphcode}[1]{{\small\lstinline[basicstyle=\small\itshape\ttfamily\color{Maroon},breaklines=true]!#1!}}
\newcommand{\codebox}[1]{\begin{lstlisting}[columns=fullflexible,breaklines=true,postbreak=\mbox{\textcolor{gray}{$\hookrightarrow$}\space}]#1\end{lstlisting}}

%\newfloat{algorithm}{ht}{aux0}              % Algorithmus-Umgebung
%\floatname{algorithm}{Code-Abschnitt}
\newcommand{\anm}[1]{\textcolor{blue}{#1}}
\def\bigA{\mathop{\mathrm{A}}}

%----------------------Funktionen, Zeichen----------------------
\def\det{\hbox{det} \,}
\def\spn{\hbox{span} \,}
\def\div{\hbox{div} \,}
\def\grad{\hbox{grad} \,}
\def\supp{\hbox{supp} \,}
\def\cof{\hbox{cof} \,}
\def\tr{\hbox{tr} \,}
\def\sym{\hbox{sym} \,}
\def\diag{\hbox{diag} \,}
\def\dyad{\otimes}
%\def\spur{\hbox{\textup{spur}} \,}
\DeclareMathOperator{\spur}{spur}
\newcommand{\stern}[1] {\overset{*}{#1}}        %Sternchen auf Buchstabe
\def\tstern{\stern{t}}
\def\dV{\d V}
\def\qed{\begin{flushright}$\square$\end{flushright}}
%\renewcommand{\grqq}{\grqq\,}
\def\rpsi{\textcolor{red}{\hat{\psi}}}
\def\Dcon{\mathcal{D}_{con}}
\def\Dloc{\mathcal{D}_{loc}}
\def\D{\mathcal{D}}
\def\E{\mathbb{E}} % C2 domain of elasticity
\def\bbC{\mathbb{C}} % elasticity tensor
\def\bbI{\mathbb{I}} % identity tensor
\def\P{\mathcal{P}} % C2 domain of elasticity
\def\G{G} % C2 domain of elasticity
\def\dt{d\noexpand\mkern-1mu t}

%----------------------Ableitungen------------------------

%Ableitungen mit \d 
\makeatletter
\def\d{\futurelet\next\start@i}\def\start@i{\ifx\next\bgroup\expandafter\abl@\else\expandafter\abl@d\fi}\def\abl@#1{\def\tempa{#1}\futurelet\next\abl@i}\def\abl@i{\ifx\next\bgroup\expandafter\abl@ii\else\expandafter\abl@a\fi}\def\abl@ii#1{\def\tempb{#1}\futurelet\next\abl@iii}\def\abl@iii{\ifx\next\bgroup\expandafter\abl@c\else\expandafter\abl@b\fi}
\def\abl@d{\mathrm{d}}                                          % keine Argumente
\def\abl@a{\ds\frac{\mathrm{d}}{\mathrm{d}\tempa}\,}            % 1 Argument \d{x} -> d/dx
\def\abl@b{\ds\frac{\mathrm{d}\tempa}{\mathrm{d}\tempb}\,}  % 2 Argumente: \d{f}{x} -> df/dx
\def\abl@c#1{\ds\frac{\mathrm{d}^{#1} {\tempa}}{\mathrm{d} {\tempb}^{#1}}\,}        % 3 Argumente: \d{f}{x}{2} -> d^2f/dx^2

%partielle Ableitungen mit \p
\def\p{\futurelet\next\startp@i}\def\startp@i{\ifx\next\bgroup\expandafter\pabl@\else\expandafter\pabl@d\fi}\def\pabl@#1{\def\tempa{#1}\futurelet\next\pabl@i}\def\pabl@i{\ifx\next\bgroup\expandafter\pabl@ii\else\expandafter\pabl@a\fi}\def\pabl@ii#1{\def\tempb{#1}\futurelet\next\pabl@iii}\def\pabl@iii{\ifx\next\bgroup\expandafter\pabl@c\else\expandafter\pabl@b\fi}
\def\pabl@d{\partial}                                           % keine Argumente
\def\pabl@a{\ds\frac{\partial}{\partial\tempa}\,}           % 1 Argument \d{x} -> d/dx
\def\pabl@b{\ds\frac{\partial\tempa}{\partial\tempb}\,} % 2 Argumente: \d{f}{x} -> df/dx
\def\pabl@c#1{\ds\frac{\partial^{#1} {\tempa}}{\partial {\tempb}^{#1}}\,}       % 3 Argumente: \d{f}{x}{2} -> d^2f/dx^2
\makeatother

%i-ter Ableitungsoperator
\newcommand{\dd}[2]{\ds\frac{\mathrm{d}^{#2}}{\mathrm{d}{#1}^{#2}}\,}   %\dd{x}{5} -> d^5/dx^5
\newcommand{\pp}[2]{\ds\frac{\partial^{#2}}{\partial{#1}^{#2}}\,}       %\pp{x}{5} -> d^5/dx^5 (partiell)

%----------------------Buchstaben, Räume----------------------
\def\eps{\varepsilon}
\def\N{\mathbb{N}}  %nat. Zahlen
\def\Z{\mathbb{Z}}  %ganze Zahlen
\def\Q{\mathbb{Q}}  %rat. Zahlen
\def\R{\mathbb{R}}  %reelle Zahlen
\def\C{\mathbb{C}}  %komplexe Zahlen
\def\P{\mathcal{P}} %Potenzmenge, Polynome
\def\T{\mathcal{T}} %Triangulierung
\def\Oe{\overset{..}{O}}    %Menge von 2013_12_04
\def\DD{\mathcal{D}} % Differentialoperator

\renewcommand{\i}[2]{\ds\int\limits_{#1}^{#2}} %Integral, %TODO_Lorin:das überschreibt "interpolierende" \I, %FIX_Benni: zweimal kleiner Buchstabe (Großbuchstaben sind eher für Räume)
\renewcommand{\s}[2]{\ds\sum\limits_{#1}^{#2}} %Summe %EDIT_Georg: mit renewcommand hat's nicht compiliert, deshalb jetzt newcommand


\renewcommand{\O}{\mathcal{O}}      %O-Notation
\renewcommand{\o}{o}
\newcommand{\CC}{\mathcal{C}}       %Raum der stetig diff.baren Fkt
\renewcommand{\L}{\mathcal{L}}      %Raum der Lebesgue-int.baren Fkt
\newcommand{\W}{\mathcal{W}}
\newcommand{\Lloc}{\L^1_{\text{loc}}}
\newcommand{\Cabh}{\mathrm{C}}      %Abhängigkeitskegel
\newcommand{\Sabh}{\mathrm{S}}      %zum Abhängigkeitskegel gehörendes S

% Maßeinheiten
\newcommand{\cm}{\,\mathrm{cm}}
\newcommand{\m}{\,\mathrm{m}}
\newcommand{\Npcm}{\,\mathrm{N/cm}}
\newcommand{\Npm}{\,\mathrm{N/m}}
\newcommand{\Npmm}{\,\mathrm{N/m^2}}
\newcommand{\NN}{\,\mathrm{N}}

%---------------------fette Buchstaben------------------------
\newcommand{\bfa}{\textbf{a}}
\newcommand{\bfb}{\textbf{b}}
\newcommand{\bfc}{\textbf{c}}
\newcommand{\bfd}{\textbf{d}}
\newcommand{\bfe}{\textbf{e}}
\newcommand{\bff}{\textbf{f}}
\newcommand{\bfg}{\textbf{g}}
\newcommand{\bfh}{\textbf{h}}
\newcommand{\bfi}{\textbf{i}}
\newcommand{\bfj}{\textbf{j}}
\newcommand{\bfk}{\textbf{k}}
\newcommand{\bfl}{\textbf{l}}
\newcommand{\bfm}{\textbf{m}}
\newcommand{\bfn}{\textbf{n}}
\newcommand{\bfo}{\textbf{o}}
\newcommand{\bfp}{\textbf{p}}
\newcommand{\bfq}{\textbf{q}}
\newcommand{\bfr}{\textbf{r}}
\newcommand{\bfs}{\textbf{s}}
\newcommand{\bft}{\textbf{t}}
\newcommand{\bfu}{\textbf{u}}
\newcommand{\bfv}{\textbf{v}}
\newcommand{\bfw}{\textbf{w}}
\newcommand{\bfx}{\textbf{x}}
\newcommand{\bfy}{\textbf{y}}
\newcommand{\bfz}{\textbf{z}}
\newcommand{\bfA}{\textbf{A}}
\newcommand{\bfB}{\textbf{B}}
\newcommand{\bfC}{\textbf{C}}
\newcommand{\bfD}{\textbf{D}}
\newcommand{\bfE}{\textbf{E}}
\newcommand{\bfF}{\textbf{F}}
\newcommand{\bfG}{\textbf{G}}
\newcommand{\bfH}{\textbf{H}}
\newcommand{\bfI}{\textbf{I}}
\newcommand{\bfJ}{\textbf{J}}
\newcommand{\bfK}{\textbf{K}}
\newcommand{\bfL}{\textbf{L}}
\newcommand{\bfM}{\textbf{M}}
\newcommand{\bfN}{\textbf{N}}
\newcommand{\bfO}{\textbf{O}}
\newcommand{\bfP}{\textbf{P}}
\newcommand{\bfQ}{\textbf{Q}}
\newcommand{\bfR}{\textbf{R}}
\newcommand{\bfS}{\textbf{S}}
\newcommand{\bfT}{\textbf{T}}
\newcommand{\bfU}{\textbf{U}}
\newcommand{\bfV}{\textbf{V}}
\newcommand{\bfW}{\textbf{W}}
\newcommand{\bfX}{\textbf{X}}
\newcommand{\bfY}{\textbf{Y}}
\newcommand{\bfZ}{\textbf{Z}}
\newcommand{\bfzero}{\textbf{0}}

\newcommand{\bfalpha}{\boldsymbol{\alpha}}
\newcommand{\bfbeta}{\boldsymbol{\beta}}
\newcommand{\bfgamma}{\boldsymbol{\gamma}}
\newcommand{\bfdelta}{\boldsymbol{\delta}}
\newcommand{\bfeps}{\boldsymbol{\eps}}
\newcommand{\bftheta}{\boldsymbol{\theta}}
\newcommand{\bfeta}{\boldsymbol{\eta}}
\newcommand{\bfphi}{\boldsymbol{\phi}}
\newcommand{\bfvarphi}{\boldsymbol{\varphi}}
\newcommand{\bfpsi}{\boldsymbol{\psi}}
\newcommand{\bfsigma}{\boldsymbol{\sigma}}
\newcommand{\bfPi}{\boldsymbol{\Pi}}
\newcommand{\bfXi}{\boldsymbol{\Xi}}
\newcommand{\bfxi}{\boldsymbol{\xi}}
\newcommand{\bfchi}{\boldsymbol{\chi}}
\newcommand{\bfChi}{\boldsymbol{\chi}}
\newcommand{\bfzeta}{\boldsymbol{\zeta}}
\newcommand{\bfmu}{\boldsymbol{\mu}}
\newcommand{\bflambda}{\boldsymbol{\lambda}}
