\documentclass[fleqn,reqno,a4paper,parskip=half]{scrartcl}
%\usepackage{showkeys}      % zeigt label-Bezeichner an

%%%%%%%%%%%%%%%%   Pakete   %%%%%%%%%%%%%%%%%%

\usepackage{ifxetex}
\ifxetex                % Pakete für XeLaTex / XeTex
    
    \usepackage{fontspec}
    \defaultfontfeatures{Mapping=tex-text}
    \usepackage{unicode-math}
    
    %------------- Schriftarten: ------------------------------
    %\setmathfont{xits-math.otf}
    %\setmathfont{latinmodern-math.otf}
    %\setmathfont{texgyrepagella-math.otf}
    %\setmathfont{Asana-Math.otf}
    
\else                   % Befehle für pdflatex
%   \usepackage[utf8]{inputenc}


\usepackage[mathletters]{ucs} %direkt griechisches im Mathe modus
\usepackage[utf8x]{inputenc}
%\usepackage[T1]{fontenc}
%\usepackage{times}

%TODO_Lorin: besser?
   %\usepackage{uniinput}       % für Unicode-Zeichen, wird momentan nicht verwendet, deshalb auskommentiert by Benni
\fi


\usepackage[ngerman, english]{babel} % zuletzt genannte Sprache ist aktiv
%\usepackage{ngerman}
\usepackage[tbtags,sumlimits,intlimits,namelimits]{amsmath}

%\usepackage{amsfonts}
\usepackage{amssymb}
\usepackage{bbm}
\usepackage{ulem}
\usepackage{tikz}
\usepackage{pgf}
\usepackage{ifpdf}
\usepackage{color}
\usepackage{esint}
\usepackage{framed}
%\usepackage{harmony}   auskommentiert by Georg, da beim compilieren
%                       "harmony.sty not found"
%\usepackage[colorlinks=true,linkcolor=black,citecolor=black,urlcolor=black]{hyperref}  % print
\usepackage[colorlinks=true,linkcolor=blue,citecolor=blue]{hyperref}    % web
\usepackage[top=2.3cm, bottom=3.45cm, left=2.3cm, right=2.3cm]{geometry}
%\numberwithin{equation}{section}
\usepackage{chngcntr}
\counterwithin*{section}{part}
%\graphicspath{{images/png/}{images/}}        % Pfad, in dem sich Grafikdateien befinden
%\usepackage{subfigure}          % Unterbilder, deprecated
%\usepackage(subfig}

\usepackage[all]{hypcap}
\usepackage{cite}           % Literatur
\usepackage{graphicx}       % Bilder in Tabellen
\usepackage{float}          % eigene Float-Umgebungen, H-Option, um Bilder an der aktuellen Stelle anzuzeigen
\usepackage{caption}
\usepackage{subcaption,array}
%\usepackage{subcaption}

\restylefloat{figure}       % Bilder an der Stelle, wo sie eingebunden werden
\usepackage{multirow}
\usepackage{listings}       % Darstellung von Source-Code
\usepackage{framed}         % Rahmen um Text
\usepackage{mdframed}       % Rahmen um Text und Gleichungen
%\usepackage{arydshln}      % gestrichelte Linie in Tabelle mit \hdashline
\usepackage{dirtytalk}          % \say{...} erzeugt (deutsche) Anführungszeichen

\usepackage{tipa}
\usepackage{transparent}    % needed for inkscape generated pdf_tex files
\usepackage{multicol}       % multiple columns
\usepackage{moreverb}       % verbatimwrite
\usepackage{verbatimbox}    % \begin{verbbox}
\usepackage{booktabs}
\usepackage{morefloats}
\usepackage{cleveref}
\usepackage{mathrsfs}       % mathscr

\usepackage{multimedia}     % \movie

\newsavebox\lstbox
\mdfdefinestyle{MyFrame}{%
    innertopmargin=0pt,
    innerbottommargin=10pt,
    innerrightmargin=20pt,
    innerleftmargin=20pt}

\definecolor{darkgreen}{HTML}{009900}
    
% settings for algorithm
\lstset{literate=%
    {Ö}{{\"O}}1
    {Ä}{{\"A}}1
    {Ü}{{\"U}}1
    {ß}{{\ss}}1
    {ü}{{\"u}}1
    {ä}{{\"a}}1
    {ö}{{\"o}}1
    {⇐}{{$\leftarrow$}}1
    {>=}{{$\geq$}}1
    {~}{{\textasciitilde}}1,  
  language=C++,
  numbers=none,
  numberstyle=\tiny, 
  basicstyle=\small, %  print  whole  listing  small
  morekeywords={elif,do,end,then,proc,local,Eingabe,Ausgabe,alignof,loop,each},
  deletekeywords={new},
  columns=flexible,   % alignment
  tabsize=2,    % size of tabs
  keepspaces,
  gobble=2,    % remove 2 characters at begin of each line
  mathescape    % wandle $$ in latex um
}

% Versuche stärker, Abbildungen dort einzubinden, wo sie definiert wurden
\renewcommand{\topfraction}{.85}      % Anteil, den floats auf einer Seite von oben her einnehmen dürfen
\renewcommand{\bottomfraction}{.7}    % Anteil, den floats auf einer Seite von unten her einnehmen dürfen
\renewcommand{\textfraction}{.15}       % Anteil der Seite, der mind. für Text zur Verfügung steht
\renewcommand{\floatpagefraction}{.66}  % Anteil der Seite, der belegt sein muss, bevor eine weitere Seite angelegt wird
\setcounter{topnumber}{9}               % maximale Anzahl floats, die im oberen Bereich der Seite sein dürfen
\setcounter{bottomnumber}{9}            % maximale Anzahl floats, die im unteren Bereich der Seite sein dürfen
    
\newcommand{\bild}[3]{%
    \begin{figure}%
        \centering%
        \def\svgwidth{#2}%
        \input{images/#1.pdf_tex}%
        \caption{#3}%
        \label{fig:#1}%
    \end{figure}%
}

\newcommand{\subfig}[3]{%
    \begin{subfigure}[b]{#2}%
        \centering%
        \def\svgwidth{#2}%
        \input{images/#1.pdf_tex}%
        \caption{#3}%
        \label{fig:#1}%
    \end{subfigure}%
}

\newcommand{\subfigpng}[3]{%
    \begin{subfigure}[t]{#2}%
        \centering%
        \includegraphics[width=#2]{images/#1.png}%
        \caption{#3}%
        \label{fig:#1}%
    \end{subfigure}%
}
\newcommand{\subfigpngheight}[4]{%
    \begin{subfigure}[t]{#2}%
        \centering%
        \includegraphics[height=#3]{images/#1.png}%
        \caption{#4}%
        \label{fig:#1}%
    \end{subfigure}%
}

\newcommand{\subfigpdf}[3]{%
    \begin{subfigure}[b]{#2}%
        \centering%
        \includegraphics[width=#2]{images/#1.pdf}%
        \caption{#3}%
        \label{fig:#1}%
    \end{subfigure}%
}

\newcommand{\subfigsvg}[3]{%
    \begin{subfigure}[b]{#2}%
        \centering%
        \includegraphics[width=#2]{images/#1.svg}%
        \caption{#3}%
        \label{fig:#1}%
    \end{subfigure}%
}

\newcommand{\bildpng}[3]{%
    \begin{figure}[ht]%
        \centering%
        \includegraphics[width=#2]{images/#1.png}%
        \caption{#3}%
        \label{fig:#1}%
    \end{figure}%
}
\newcommand{\bildsvg}[3]{%
    \begin{figure}[ht]%
        \centering%
        \includegraphics[width=#2]{images/#1.svg}%
        \caption{#3}%
        \label{fig:#1}%
    \end{figure}%
}

\newcommand{\bildpdf}[3]{%
    \begin{figure}%
        \centering%
        \includegraphics[width=#2]{images/#1.pdf}%
        \caption{#3}%
        \label{fig:#1}%
    \end{figure}%
}

%%%%%%%%%%%%%%%%   Abkürzungen   %%%%%%%%%%%%%%%%%%

%----------------------Umgebungen----------------------
\def\beqno{\begin{equation}}
\def\eeqno{\end{equation}}
\def\beq{\begin{equation*}}
\def\eeq{\end{equation*}}
\def\ba#1{\begin{array}{#1}}
\def\ea{\end{array}}
\def\mat#1{\left(\begin{matrix}#1\end{matrix}\right)}   % added by Georg

\newcommand{\Name}[1]   {\textit{#1}\/}                 % Eigennamen kursiv
\renewcommand{\emph}[1]{\textit{#1}\/}
\def\clap#1{\hbox  to  0pt{\hss#1\hss}}                 % für underbrace
\def\mathclap{\mathpalette\mathclapinternal}
\def\mathclapinternal#1#2{\clap{$\mathsurround=0pt#1{#2}$}}
\newcommand{\ub}[2]{\underbrace{#1}_{\mathclap{#2}}}    
\newcommand{\ds}{\displaystyle}                         % displaystyle
\newcommand{\scr}{\scriptstyle}
\renewcommand{\dfrac}[2]{\ds\frac{\ds{#1}}{\ds{#2}}\,}  % nach Bruch Abstand
%\newcommand{\code}[1]{{\small\lstinline[columns=fixed]!#1!}}

\usepackage{setspace}
\newcommand{\code}[1]{{\small\lstinline[basicstyle=\footnotesize\ttfamily]!#1!}}
\newfloat{algorithm}{ht}{aux0}              % Algorithmus-Umgebung
\floatname{algorithm}{Code-Abschnitt}
\newcommand{\anm}[1]{\textcolor{blue}{#1}}
\def\bigA{\mathop{\mathrm{A}}}

%----------------------Funktionen, Zeichen----------------------
\def\det{\hbox{det} \,}
\def\span{\hbox{span} \,}
\def\div{\hbox{div} \,}
\def\grad{\hbox{grad} \,}
\def\supp{\hbox{supp} \,}
\def\tr{\hbox{tr} \,}
\def\dyad{\otimes}
%\def\spur{\hbox{\textup{spur}} \,}
\DeclareMathOperator{\spur}{spur}
\newcommand{\stern}[1] {\overset{*}{#1}}        %Sternchen auf Buchstabe
\def\tstern{\stern{t}}
\def\dV{\d V}
\def\qed{\begin{flushright}$\square$\end{flushright}}
%\renewcommand{\grqq}{\grqq\,}
\def\rpsi{\textcolor{red}{\hat{\psi}}}
\def\Dcon{\mathcal{D}_{con}}
\def\Dloc{\mathcal{D}_{loc}}
\def\D{\mathcal{D}}
\def\E{\mathbb{E}} % C2 domain of elasticity
\def\bbC{\mathbb{C}} % elasticity tensor
\def\bbI{\mathbb{I}} % identity tensor
\def\P{\mathcal{P}} % C2 domain of elasticity
\def\G{G} % C2 domain of elasticity

%----------------------Ableitungen------------------------

%Ableitungen mit \d 
\makeatletter
\def\d{\futurelet\next\start@i}\def\start@i{\ifx\next\bgroup\expandafter\abl@\else\expandafter\abl@d\fi}\def\abl@#1{\def\tempa{#1}\futurelet\next\abl@i}\def\abl@i{\ifx\next\bgroup\expandafter\abl@ii\else\expandafter\abl@a\fi}\def\abl@ii#1{\def\tempb{#1}\futurelet\next\abl@iii}\def\abl@iii{\ifx\next\bgroup\expandafter\abl@c\else\expandafter\abl@b\fi}
\def\abl@d{\mathrm{d}}                                          % keine Argumente
\def\abl@a{\ds\frac{\mathrm{d}}{\mathrm{d}\tempa}\,}            % 1 Argument \d{x} -> d/dx
\def\abl@b{\ds\frac{\mathrm{d}\tempa}{\mathrm{d}\tempb}\,}  % 2 Argumente: \d{f}{x} -> df/dx
\def\abl@c#1{\ds\frac{\mathrm{d}^{#1} {\tempa}}{\mathrm{d} {\tempb}^{#1}}\,}        % 3 Argumente: \d{f}{x}{2} -> d^2f/dx^2

%partielle Ableitungen mit \p
\def\p{\futurelet\next\startp@i}\def\startp@i{\ifx\next\bgroup\expandafter\pabl@\else\expandafter\pabl@d\fi}\def\pabl@#1{\def\tempa{#1}\futurelet\next\pabl@i}\def\pabl@i{\ifx\next\bgroup\expandafter\pabl@ii\else\expandafter\pabl@a\fi}\def\pabl@ii#1{\def\tempb{#1}\futurelet\next\pabl@iii}\def\pabl@iii{\ifx\next\bgroup\expandafter\pabl@c\else\expandafter\pabl@b\fi}
\def\pabl@d{\partial}                                           % keine Argumente
\def\pabl@a{\ds\frac{\partial}{\partial\tempa}\,}           % 1 Argument \d{x} -> d/dx
\def\pabl@b{\ds\frac{\partial\tempa}{\partial\tempb}\,} % 2 Argumente: \d{f}{x} -> df/dx
\def\pabl@c#1{\ds\frac{\partial^{#1} {\tempa}}{\partial {\tempb}^{#1}}\,}       % 3 Argumente: \d{f}{x}{2} -> d^2f/dx^2
\makeatother

%i-ter Ableitungsoperator
\newcommand{\dd}[2]{\ds\frac{\mathrm{d}^{#2}}{\mathrm{d}{#1}^{#2}}\,}   %\dd{x}{5} -> d^5/dx^5
\newcommand{\pp}[2]{\ds\frac{\partial^{#2}}{\partial{#1}^{#2}}\,}       %\pp{x}{5} -> d^5/dx^5 (partiell)

%----------------------Buchstaben, Räume----------------------
\def\eps{\varepsilon}
\def\N{\mathbb{N}}  %nat. Zahlen
\def\Z{\mathbb{Z}}  %ganze Zahlen
\def\Q{\mathbb{Q}}  %rat. Zahlen
\def\R{\mathbb{R}}  %reelle Zahlen
\def\C{\mathbb{C}}  %komplexe Zahlen
\def\P{\mathcal{P}} %Potenzmenge, Polynome
\def\T{\mathcal{T}} %Triangulierung
\def\Oe{\overset{..}{O}}    %Menge von 2013_12_04
\def\DD{\mathcal{D}} % Differentialoperator

\renewcommand{\i}[2]{\ds\int\limits_{#1}^{#2}} %Integral, %TODO_Lorin:das überschreibt "interpolierende" \I, %FIX_Benni: zweimal kleiner Buchstabe (Großbuchstaben sind eher für Räume)
\renewcommand{\s}[2]{\ds\sum\limits_{#1}^{#2}} %Summe %EDIT_Georg: mit renewcommand hat's nicht compiliert, deshalb jetzt newcommand


\renewcommand{\O}{\mathcal{O}}      %O-Notation
\renewcommand{\o}{o}
\newcommand{\CC}{\mathcal{C}}       %Raum der stetig diff.baren Fkt
\renewcommand{\L}{\mathcal{L}}      %Raum der Lebesgue-int.baren Fkt
\newcommand{\W}{\mathcal{W}}
\newcommand{\Lloc}{\L^1_{\text{loc}}}
\newcommand{\Cabh}{\mathrm{C}}      %Abhängigkeitskegel
\newcommand{\Sabh}{\mathrm{S}}      %zum Abhängigkeitskegel gehörendes S

% Maßeinheiten
\newcommand{\cm}{\,\mathrm{cm}}
\newcommand{\m}{\,\mathrm{m}}
\newcommand{\Npcm}{\,\mathrm{N/cm}}
\newcommand{\Npm}{\,\mathrm{N/m}}
\newcommand{\Npmm}{\,\mathrm{N/m^2}}
\newcommand{\NN}{\,\mathrm{N}}

%---------------------fette Buchstaben------------------------
\newcommand{\bfa}{\textbf{a}}
\newcommand{\bfb}{\textbf{b}}
\newcommand{\bfc}{\textbf{c}}
\newcommand{\bfd}{\textbf{d}}
\newcommand{\bfe}{\textbf{e}}
\newcommand{\bff}{\textbf{f}}
\newcommand{\bfg}{\textbf{g}}
\newcommand{\bfh}{\textbf{h}}
\newcommand{\bfi}{\textbf{i}}
\newcommand{\bfj}{\textbf{j}}
\newcommand{\bfk}{\textbf{k}}
\newcommand{\bfl}{\textbf{l}}
\newcommand{\bfm}{\textbf{m}}
\newcommand{\bfn}{\textbf{n}}
\newcommand{\bfo}{\textbf{o}}
\newcommand{\bfp}{\textbf{p}}
\newcommand{\bfq}{\textbf{q}}
\newcommand{\bfr}{\textbf{r}}
\newcommand{\bfs}{\textbf{s}}
\newcommand{\bft}{\textbf{t}}
\newcommand{\bfu}{\textbf{u}}
\newcommand{\bfv}{\textbf{v}}
\newcommand{\bfw}{\textbf{w}}
\newcommand{\bfx}{\textbf{x}}
\newcommand{\bfy}{\textbf{y}}
\newcommand{\bfz}{\textbf{z}}
\newcommand{\bfA}{\textbf{A}}
\newcommand{\bfB}{\textbf{B}}
\newcommand{\bfC}{\textbf{C}}
\newcommand{\bfD}{\textbf{D}}
\newcommand{\bfE}{\textbf{E}}
\newcommand{\bfF}{\textbf{F}}
\newcommand{\bfG}{\textbf{G}}
\newcommand{\bfH}{\textbf{H}}
\newcommand{\bfI}{\textbf{I}}
\newcommand{\bfJ}{\textbf{J}}
\newcommand{\bfK}{\textbf{K}}
\newcommand{\bfL}{\textbf{L}}
\newcommand{\bfM}{\textbf{M}}
\newcommand{\bfN}{\textbf{N}}
\newcommand{\bfO}{\textbf{O}}
\newcommand{\bfP}{\textbf{P}}
\newcommand{\bfQ}{\textbf{Q}}
\newcommand{\bfR}{\textbf{R}}
\newcommand{\bfS}{\textbf{S}}
\newcommand{\bfT}{\textbf{T}}
\newcommand{\bfU}{\textbf{U}}
\newcommand{\bfV}{\textbf{V}}
\newcommand{\bfW}{\textbf{W}}
\newcommand{\bfX}{\textbf{X}}
\newcommand{\bfY}{\textbf{Y}}
\newcommand{\bfZ}{\textbf{Z}}
\newcommand{\bfzero}{\textbf{0}}

\newcommand{\bfeps}{\boldsymbol{\eps}}
\newcommand{\bfsigma}{\boldsymbol{\sigma}}
\newcommand{\bfPi}{\boldsymbol{\Pi}}
\newcommand{\bfXi}{\boldsymbol{\Xi}}
\newcommand{\bfxi}{\boldsymbol{\xi}}
\newcommand{\bfzeta}{\boldsymbol{\zeta}}
\newcommand{\bfmu}{\boldsymbol{\mu}}


\graphicspath{
{images/png/}{images/}{images/plots/}
}


\begin{document}

%\setcounter{tocdepth}{2}
%\tableofcontents
%\newpage

\section{Laplace equation}
\label{chap:laplace}

For a computational domain $\Omega\subset \R^d$ the Laplace equation reads
\begin{equation}\label{eq:laplace}
  \begin{array}{ll}
    Δu = 0 \quad \text{on }\Omega.
  \end{array}
\end{equation}
A classical solution $u: \Omega \to \R$ fulfills \eqref{eq:laplace}. For a unique solution also boundary conditions have to be specified, e.g.
\begin{equation}
  \begin{array}{rcll}
    ∇u(\bfx) \cdot \bfn &=& 0 \quad &\text{on } \Gamma_N,\\[4mm]
    u(\bfx) &=& u_0(\bfx) \quad &\text{on } \Gamma_D,
  \end{array}
\end{equation}
where the homogeneous Neumann-type boundary conditions for $\bfx \in \Gamma_N$ set the flux over the boundary in normal direction $\bfn$ to zero and the Dirichlet-type boundary conditions on $\Gamma_D$ prescribe a value for $u$ on the boundary.

\subsection{Finite Element formulation}

By multiplication of a testfunction $\phi\in H^1_0(\Omega)$ and integration follows the weak formulation of \eqref{eq:laplace}:
\begin{equation}
  \begin{array}{ll}
    \ds\int_{\Omega}Δu\,\phi\,\d \bfx = 0 \quad \forall \phi\in H^1_0(\Omega)
  \end{array}
\end{equation}
For a definition of $H^1_0$ see section \ref{sec:hilbert}.

The Laplace operator can be written as $Δu=∇\cdot(∇u)$. Applying divergence theorem in form of \eqref{eq:gauss1} with $f=\phi$ and $\bfF=∇u$ yields
\begin{equation}
  \begin{array}{ll}
    -\ds\int_{\Omega}∇u \cdot ∇\phi \,\d \bfx + \ds\int_{\p \Omega} (\phi\,∇u)\cdot\bfn\,\d \bfx  = 0 \quad \forall \phi\in H^1_0(\Omega)
  \end{array}
\end{equation}
Because $\phi$ is zero on the boundary, the second term vanishes:
\begin{equation}\label{eq:laplace_weak}
  \begin{array}{ll}
    -\ds\int_{\Omega}∇u \cdot ∇\phi \,\d \bfx = 0 \quad \forall \phi\in H^1_0(\Omega)
  \end{array}
\end{equation}

Now we have to specify a finite-dimensional ansatz space to choose the solution function from. We do this by specifying a basis and take the span of it: $V:=\span\{\phi_1, \dots \phi_n\}$.

The numerical solution is given by
\begin{equation}
  \begin{array}{ll}
    u_h(\bfx) = \s{i=1}{N} u_i\,\phi_i(\bfx).
  \end{array}
\end{equation}
We also take $V$ as the space of testfunctions.
Plugging this into \eqref{eq:laplace_weak} yields
\begin{equation}\label{eq:laplace_discretized}
  \begin{array}{ll}
    -\s{i=1}{N} u_i \ds\int_{\Omega}∇\phi_i\cdot ∇\phi_j\,\d\bfx = 0 \quad \text{for }j=1,\dots,N.
  \end{array}
\end{equation}
The minus sign is kept for similarity with other problems.

A reasonable choice of ansatz functions are functions that have limited support. We discretize the domain $\Omega$ by Finite Elements $\Omega_e$,
\begin{equation}
  \begin{array}{ll}
    \Omega = \overset{M}{\underset{e=1}{\bigcup}} \,\Omega_e = \Omega_1 \dot{\cup} \cdots \dot{\cup} \Omega_M,
  \end{array}
\end{equation} and define nodes with global indices $N(e)$ on each element $e$. Interpolating ansatz functions are now chosen such that they have the value 1 at only one node and the value 0 at all other nodes. The support is contained just within the elements that are adjacent to the node where the function is 1.

\subsection{Ansatz functions}
A simple choice that fulfills the requirements are first-order Lagrange functions $L_{i,p},p=1$ which are defined a follows for $d=1$ and depicted in \cref{fig:lagrange}.
\begin{equation}
  \begin{array}{ll}
    \varphi_i: [0,1] \to \R,\quad
    \varphi_1(x) = L_{1,1}(x) := 1-x, \qquad \varphi_2(x) = L_{2,1}(x) := x
  \end{array}
\end{equation}
For higher dimensions they are composed by a tensor product ansatz.
\begin{equation}
  \begin{array}{ll}
    \varphi_i(\bfx) = \bfL_{i}(\bfx) := \prod\limits_{k=1}^{d} L_{j,1}(x_k)
  \end{array}
\end{equation}
The local numbering of the ansatz functions of an element proceeds fastest in the first dimension then in the second and so on as shown in \cref{fig:element1}

\begin{figure}
  \centering
  \subfig{element1}{4cm}{Numbering and element coordinate system for a 2D first-order Lagrange element}\,
  \subfig{element2}{4.5cm}{Arbitrarily shaped element}
  \,
  \subfigpdf{lagrange}{6cm}{first order Lagrange ansatz functions}
  \caption{2D first-order Lagrange element}
  \label{fig:2d-lagrange}
\end{figure}

\subsection{Transformation of integration domain}
The definition of the ansatz functions was in parameter space, i.e. on the unit interval $[0,1]^d$. The corresponding coordinate system is $\bfxi = \{\xi_1, \dots \xi_d\}$. However, integration over the elements $\Omega_e$ of the computational domain is required. The node coordinates which define the elements are given in the global coordinate system $\bfx = \{x_1, \dots x_d\}$. A mapping from $\bfxi$ to $\bfx$ can be performed using multi-linear interpolation between the nodal coordinates $\bfx^i$:
\begin{equation}\label{eq:multilagrange}
  \begin{array}{ll}
    \bfx(\bfxi) = \Phi(\bfxi) := \ds\sum\limits_{i} \bfL_i(\bfxi)\,\bfx^i.
  \end{array}
\end{equation}
Note that again Lagrange functions of first order appear, but this is part of the parameter space to global space mapping and independent of the choosen ansatz functions. For 1D and 2D problems Eq.~\eqref{eq:multilagrange} can be written out as:
\begin{equation}\label{eq:fe_phi}
  \begin{array}{ll}
    \text{1D:}\quad
    \Phi(\xi_1) = (1-\xi_1)\,\bfx^1 + \xi_1\,\bfx^2\\[4mm]
    \text{2D:}\quad
    \Phi(\bfxi) = (1-\xi_1)\,(1-\xi_2)\,\bfx^1 + \xi_1\,(1-\xi_2)\,\bfx^2 + (1-\xi_1)\,\xi_2\,\bfx^3 + \xi_1\,\xi_2\,\bfx^4.
  \end{array}
\end{equation}
The node numbering and coordinate frames are defined by \cref{fig:element2}.

Starting from \eqref{eq:laplace_discretized} we now plug in the Lagrange ansatz functions for $\phi$. Then the respective functions only have to be integrated over the elements where they are defined.
We get
\begin{equation}
  \begin{array}{ll}
     -\s{e=1}{M} \sum_{i\in N(e)} u_i \ds\int_{\Omega_e} ∇\phi_i(\bfx)\cdot ∇\phi_j(\bfx)\,\d\bfx = 0 \quad \text{for }j=1,\dots,N,
  \end{array}
\end{equation}
where the sum over $i\in N(e)$ is over the nodes of element $e$.
%
We transform the integration domain from global to local coordinate frame using  \eqref{eq:integration_transformation_dd} and get:
\begin{equation}\label{eq:fe_integral}
  \begin{array}{ll}
     -\s{e=1}{M} \sum_{i\in N(e)} u_i \ds\int_{[0,1]^d} ∇\phi_i(\bfxi)\cdot ∇\phi_j(\bfxi)\,J_d(\bfxi)\,\d\bfxi = 0 \quad \text{for }j=1,\dots,N.
  \end{array}
\end{equation}
Note that the computational domain $\Omega\subset \R^3$ is always considered to be embedded in $\R^3$. The 1D and 2D cases where the mesh is fully contained within a 1D or 2D subspace are then a specialization of the general case. Think of the lower dimensional meshes as a curve ($d=1$) or a bended surface ($d=2$) embedded in $\R^3$.

\subsection{Evaluation of the integral term}\label{chap:integral1}
The integral in \eqref{eq:fe_integral} defines for $i$ and $j$ the entries $m_{ij}$ of the \emph{stiffness matrix} $M$.
The equation can be written in matrix form as
%
\begin{equation*}
  \begin{array}{lll}
    M\,\bfu = \bfzero,
  \end{array}
\end{equation*}
where $M$ contains the entries
\begin{equation*}
  \begin{array}{lll}
    m_{ij} = -\ds\int_{\Omega}∇\phi_i\cdot ∇\phi_j\,\d\bfx = -\s{e=1}{M} \sum_{i\in N(e)} \ds\int_{[0,1]^d} ∇\phi_i(\bfxi)\cdot ∇\phi_j(\bfxi)\,J_d(\bfxi)\,\d\bfxi
  \end{array}
\end{equation*}
and $\bfu = (u_1, \dots, u_N)^\top$ is the solution vector. Given $M$ the solution $\bfu$ is computed by an appropriate linear system solver.

The integral for $m_{i,j}$ depends via $J_d$ on the shape of the elements. In general, it has to be evaluated numerically. However, for special simple cases it can be computed analytically.
This includes scenarios in $d=1,2,3$ dimensions where the elements are on a rectilinear cartesian grid.

If the grid is arbitrary, analytical computation for 1D is still simple. For $d=2,3$ it is still possible, but involves more lengthy derivations that are usually performed using a computer algebra system such as \verb|sympy|. In this section the 1D and 2D cases are derived, the python \verb|sympy| code for 2D and 3D is contained in the \verb|doc| directory for further reference.

For all 1D meshes that are embedded in a 3D domain as well as rectangular cartesian 2D and 3D meshes the term $J_d(\bfxi)$ is constant within each element, i.e. it does not depend on $\bfxi$. In that case one can take it as a constant out of the integral.

\textbf{1D case.}
We now compute $m_{ij}$ for $d=1$. The transformation term $J_1(\xi)$
is defined as
%
\begin{equation*}
  \begin{array}{lll}
    J_1(\xi) = \Vert \Phi'(\xi)\Vert_2.
  \end{array}
\end{equation*}
Using the parametric representation of $\Phi$ given in \eqref{eq:fe_phi}, we derive
%
\begin{equation*}
  \begin{array}{lll}
    J_1(\xi) = \Vert \Phi'(\xi)\Vert_2 = \Vert \bfx^2 - \bfx^1 \Vert_2,
  \end{array}
\end{equation*}
which is the length of the element $e$ with nodes $\bfx^1$ and $\bfx^2$. We define it to be $l_e := \Vert\bfx^2-\bfx^1\Vert_2$, and thus have $J_1(\xi) = l_e$.

%We use the derivatives of the Lagrange functions,
%\begin{equation}
%  \begin{array}{ll}
%    L_{1,1}'(\xi) = -1, \quad L_{2,1}'(\xi) = 1.\\[4mm]
%  \end{array}
%\end{equation}

\textbf{2D case.}
For 2D we assume a rectangular element that lies in a $z=$constant plane with side lengths $l_{1,e}$ and $l_{2,e}$ in $\xi_1$ and $\xi_2$ directions.
The mapping from $\bfxi=(\xi_1,\xi_2)$ to $\bfx$ coordinate frame is given by
\begin{equation}
  \begin{array}{ll}
    \Phi(\bfxi) = \bfx^1 + \mat{\xi_1\,l_{1,e} \\[2mm] \xi_2\,l_{2,e}}.
  \end{array}
\end{equation}
Then we derive
\begin{equation*}
  \begin{array}{lll}
    D\Phi(\bfxi) = \mat{l_{1,e}  & 0 \\[2mm] 0 & l_{2,e}}
  \end{array}
\end{equation*}
and
\begin{equation*}
  \begin{array}{lll}
    J_2 = \sqrt{\det \big(D\Phi(\bfxi)^\top D\Phi(\bfxi)\big)} = |l_{1,e}\,l_{2,e}|.
  \end{array}
\end{equation*}
%
\textbf{3D case.}
Similar to the 2D case when a rectangular 3D grid with grid widths $l_{1,e}, l_{2,e},l_{3,e}$ is assumed, the transformation factor becomes
\begin{equation*}
  \begin{array}{lll}
    J_3 = |l_{1,e}\,l_{2,e}\,l_{3,e}|.
  \end{array}
\end{equation*}

In the presented special cases $J_d$ did not depend on the integration domain, which allows to compute the factor separately:
\begin{equation*}
  \begin{array}{lll}
    \ds\int_{\Omega} ∇\phi_i \cdot ∇\phi_j\,J_d\,\d\bfxi = J_d\,\ds\int_{\Omega} ∇\phi_i \cdot ∇\phi_j\,\d\bfxi.
  \end{array}
\end{equation*}

Now the term $-\int ∇\phi_i\cdot ∇\phi_j \,\d\xi$, remains to be computed. We compute values at the nodes and visualize them using \emph{stencil notation}. For a fixed node $i$ we compute the respective values for adjacent nodes $j$. The result for $i=j$ is underlined in the stencil, the values for adjacent nodes are placed left, right, top and bottom, in the position of the respective nodes.

We first compute element-wise stencils that state the contribution of a single element. If all elements have the same length properties, the element contributions can be summed up to get the total value at the nodes which is shown in the nodal stencils. From these stencils we can easily set up the stiffness matrix for a non-varying, equidistant mesh.

\begin{tabular}{l|l|l|l}
    dim & element contribution & node stencil\\
    \hline
    1D: &
\begin{minipage}{6cm}
  \begin{equation*}
     \left[\begin{array}{ccc}
        \underline{-1} & 1\\
    \end{array}\right] \quad 
  \end{equation*}
\end{minipage} 
    &
\begin{minipage}{6cm}
  \begin{equation*}
    \left[\begin{array}{ccc}
        1 & \underline{-2} & 1\\
    \end{array}\right]
  \end{equation*}
\end{minipage} 
     \\[4mm]
     \hline
    2D:&
\begin{minipage}{6cm}
  \begin{equation*}
    \left[
      \begin{array}{ccc}
        1/6 & 1/3 \\
        \underline{-2/3} & 1/6
      \end{array}
    \right]
  \end{equation*}
\end{minipage}  &
\begin{minipage}{6cm}
  \begin{equation*}
      \dfrac13\left[
        \begin{array}{ccc}
          1 & 1 & 1\\
          1 & \underline{-8} & 1 \\
          1 & 1 & 1
        \end{array}
      \right]
  \end{equation*}
\end{minipage}  \\[4mm]
    \hline
    3D: &
\begin{minipage}{6cm}
  \begin{equation*}
    \begin{array}{ll}
      \text{center:} &
      \left[\begin{array}{ccc}
          0 & 1/12\\
          \underline{-1/3} & 0\\
      \end{array}\right] \\[4mm]
      \text{top:}& 
      \left[\begin{array}{ccc}
          1/12 & 1/12\\
             0 & 1/12\\
      \end{array}\right]
    \end{array}
  \end{equation*}
\end{minipage} &
\begin{minipage}{6cm}
  \begin{equation*}
    \begin{array}{ll}
      \text{bottom:} &
      \dfrac1{12}
      \left[\begin{array}{ccc}
          1 & 2 & 1\\
          2 & 0 & 2\\
          1 & 2 & 1
      \end{array}\right] \\[4mm]
      \text{center:} &
      \dfrac1{12}
      \left[\begin{array}{ccc}
          2 & 0 & 2\\
          0 & \underline{-32} & 0\\
          2 & 0 & 2
      \end{array}\right] \\[4mm]
      \text{top:}& 
      \dfrac1{12}
      \left[\begin{array}{ccc}
          1 & 2 & 1\\
          2 & 0 & 2\\
          1 & 2 & 1 
      \end{array}\right]
    \end{array}  
  \end{equation*}
\end{minipage}
\end{tabular}

\subsection{Boundary Conditions}
\label{sec:bc}
The Dirichlet-type boundary condition
%
\begin{equation*}
  \begin{array}{lll}
    ∇u(\bfx)\cdot \bfn = 0 \qquad \text{on }\Gamma_N
  \end{array}
\end{equation*}
%
is satisfied automatically by the Galerkin finite element formulation. Starting from the left hand side of \eqref{eq:laplace_weak} and using Divergence theorem we get:
%
\begin{equation*}
  \begin{array}{lll}
    -\i{\Omega}{} ∇u\cdot ∇\phi \,\d \bfx = -\i{∂\Omega}{} \phi\,\big(∇u\cdot \bfn\big) \,\d \bfx + \i{\Omega}{} Δu\,\phi  \,\d \bfx = 0 \qquad ∀ \phi \in H^1_0(\Omega)
  \end{array}
\end{equation*}
%
Because $Δu = 0$ on $\Omega$ we get $∇u\cdot \bfn=0$ on the boundary.

Neumann boundary conditions can be easily considered at the discretized system. For each condition $u_i = u_{0,i}$ that enforces the degree of freedom $i$ to have the value $u_{0,i}$ we modify the linear system of equations. In the  right hand side vector we subtract from the value $f_{j}$ the product of $a_{ji}$ and the given value $u_{0,i}$ for every $j\neq i$, i.e. the new value is $\hat{f_j} = f_j - a_{ji}\,u_{0,i}$. We set $f_i = u_{0,i}$. In the matrix we zero the row and column that contain the entry $a_{ii}$, i.e. $a_{ij} = a_{ji} = 0, ∀j\neq i$ and set $a_{ii}=1$. As an example, consider the system
%
\begin{equation*}
  \begin{array}{lll}
    \mat{m_{11} & m_{12} & m_{13} \\ m_{21} & m_{22} & m_{23} \\ m_{31} & m_{32} & m_{33}}
    \mat{u_1 \\ u_2 \\ u_3} = \mat{0 \\ 0 \\ 0}
  \end{array}
\end{equation*}
with the Dirichlet boundary condition $u_3 = u_{0,3}$. The modified system then reads
%
\begin{equation*}
  \begin{array}{lll}
    \mat{m_{11} & m_{12} & 0 \\ m_{21} & m_{22} & 0 \\ 0 & 0 & 1}
    \mat{u_1 \\ u_2 \\ u_3} = \mat{-m_{13}\,u_{0,3} \\ -m_{23}\,u_{0,3} \\ u_{0,3}}.
  \end{array}
\end{equation*}

\subsection{Function spaces}
\label{sec:hilbert}
%
For the weak solutions $u$ of the problems we do not need to request $\CC^2(\Omega)$, since only the first derivatives are needed and only in a weak sense. Therefore $u\in H^1_0(\Omega)$ suffices.

The Hilbert space $H^1(\Omega)$ is the Sobolev space $\W^{1,2}(\Omega)$ which is defined using weak derivatives. The concept of weak derivatives generalizes the classical derivatives.

Let $u,v\in \Lloc(\Omega)$ and $\alpha \in \N^d_0$ a multi-index. Then $v$ is called \emph{weak derivative} of $u$ of order $\alpha$ iff
\begin{equation}
  \begin{array}{ll}
    \i{\Omega}{}u(\bfx) \D^\alpha \phi(\bfx) \,\d \bfx = (-1)^{|\alpha|} \i{\Omega}{} v(\bfx)\,\phi(\bfx)\,\d \bfx
  \end{array}
\end{equation}
for all $\phi \in \CC^\infty_0(\Omega)$. We then write $\D^\alpha u = v$. The derivative with the multi-index, $\D^\alpha$ is given by
\begin{equation}
  \begin{array}{ll}
    \D^\alpha = \dfrac{\p^{|\alpha|}}{\p^{\alpha_1}_{x_1} \cdots \p^{\alpha_d}_{x_d}}
  \end{array}
\end{equation}

If $u$ is differentable in a classical sense, the classical derivatives are also the weak derivatives. 

Now we define the \emph{Sobolev} space $\W^{1,2}(\Omega)$ (1=first order weak derivatives, 2=derivatives in $\L^2(\Omega)$)) as follows:
\begin{equation}
  \begin{array}{ll}
    \W^{1,2}(\Omega) := \{ u \in \Lloc(\Omega) \mid |\alpha| \in \N^d_0, |\alpha| \leq 1, \D^\alpha u \text{ exists}, \D^\alpha u \in \L^2(\Omega)\}.
  \end{array}
\end{equation}
With an appropriate Sobolev norm, $\W^{1,2}$ is a Banach space, i.e. complete (Cauchy series converge in it).

Together with the scalar product
\begin{equation}
  \begin{array}{ll}
    (u,v)_{H^1} := \sum\limits_{|\alpha|\leq 1} \i{\Omega}{}{\D^\alpha u(\bfx) \,\D^\alpha v(\bfx) \,\d\bfx}
  \end{array}
\end{equation}
we get the Hilbert space $H^1(\Omega) := \W^{1,2}(\Omega)$.

With $H^1_0(\Omega) := \{u \in H^1(\Omega) \mid u(\bfx) = 0 \text{ for } \bfx \in \p \Omega\}$ we denote the subspace of functions that are 0 on the boundary.

%-------------------------------------------------------------------------------------------------

\section{Poisson Equation}
The Poisson equation is a generalization of the Laplace equation and reads
%
\begin{equation*}
  \begin{array}{lll}
    Δu = f\qquad \text{on }\Omega.
  \end{array}
\end{equation*}
%
It can be subject to the same boundary conditions as Laplace equation, i.e. Neumann-type boundary conditions
%
\begin{equation*}
  \begin{array}{lll}
    ∇u(\bfx) \cdot \bfn = 0 \qquad \text{on }\Gamma_N,
  \end{array}
\end{equation*}
%
as well as Dirichlet-type boundary conditions
%
\begin{equation*}
  \begin{array}{lll}
    u(\bfx) = u_0(\bfx) \qquad \text{on }\Gamma_D.
  \end{array}
\end{equation*}
The finite element formulation proceeds similar to Chap.~\ref{chap:laplace}, multiplication of a testfunction $\phi \in H^{1}_0(\Omega)$ and integration yields
\begin{equation*}
  \begin{array}{lll}
    \ds\int_{\Omega}Δu\,\phi\,\d \bfx = \int_{\Omega} f\,\phi\,\d \bfx, \quad \forall \phi \in H^1_0(\Omega).
  \end{array}
\end{equation*}
Applying divergence theorem we get
\begin{equation*}
  \begin{array}{lll}
    -\ds\int_{\Omega} ∇u\cdot ∇\phi \,\d \bfx = \int_{\Omega} f\,\phi\,\d \bfx \quad \forall \phi \in H^{1}_0(\Omega).
  \end{array}
\end{equation*}
Like the solution $u(\bfx)$ also the right hand side $f(\bfx)$ has to be spatially discretized by a linear combination of coefficients and basis functions:
%
\begin{equation*}
  \begin{array}{lll}
    u_h(\bfx) = \s{i=1}{N}u_i\,\phi_i(\bfx),\\[4mm]
    f_h(\bfx) = \s{i=1}{N}f_i\,\phi_i(\bfx).
  \end{array}
\end{equation*}
By again choosing the space of testfunctions to be the same as the span of basis functions, ${V=\span\{\phi_1, \dots, \phi_n\}}$ we get the Galerkin formulation as
\begin{equation*}
  \begin{array}{lll}
    -\s{i=1}{N} u_i \int_\Omega ∇\phi_i\cdot ∇\phi_j \,\d\bfx = \s{i=1}{N}f_i \int_\Omega \phi_i\cdot \phi_j \,\d\bfx \quad \text{for }j = 1, \dots, N.
  \end{array}
\end{equation*}
The domain $\Omega$ is again decomposed into disjoint elements $\Omega_e, e=1,\dots, M$ and integration has only be performed over the elements where none of the basis function vanish.

The first integral term, $\int_{\Omega} ∇\phi_i\cdot ∇\phi_j\,\d\bfx$, has to be computed as described in Section \ref{chap:integral1}. How to compute the second integral term, $\int_{\Omega} \phi_i\cdot \phi_j\,\d\bfx$ is shown in the following.

Similar as before, the integration domain is transferred from element space to parameter space. For this a transformation factor $J_d$ has to be considered, which is constant for some special cases as discussed in \cref{chap:integral1}.

For the remaining integral, $\int_{\Omega} \phi_i\cdot \phi_j\,\d\bfxi$ node stencils are provided in the following table.

\begin{tabular}{l|l|l|l}
    dim & element contribution & node stencil\\
    \hline
    1D: &
\begin{minipage}{6cm}
  \begin{equation*}
     \dfrac16\left[\begin{array}{ccc}
        \underline{2} & 1\\
    \end{array}\right] \quad 
  \end{equation*}
\end{minipage} 
    &
\begin{minipage}{6cm}
  \begin{equation*}
    \dfrac16\left[\begin{array}{ccc}
        1 & \underline{4} & 1\\
    \end{array}\right]
  \end{equation*}
\end{minipage} 
     \\[4mm]
     \hline
    2D:&
\begin{minipage}{6cm}
  \begin{equation*}
    \dfrac1{36}\left[\begin{array}{ccc}
        2 & 1 \\
        \underline{4} & 2
      \end{array}
    \right]
  \end{equation*}
\end{minipage}  &
\begin{minipage}{6cm}
  \begin{equation*}
      \dfrac1{36}\left[
        \begin{array}{ccc}
          1 & 4 & 1\\
          4 & \underline{16} & 4 \\
          1 & 4 & 1
        \end{array}
      \right]
  \end{equation*}
\end{minipage}  \\[4mm]
    \hline
    3D: &
\begin{minipage}{6cm}
  \begin{equation*}
    \begin{array}{ll}
      \text{center:} &
      \dfrac1{216}\left[\begin{array}{ccc}
          4 & 2\\
          \underline{8} & 4\\
      \end{array}\right] \\[4mm]
      \text{top:}& 
      \dfrac1{216}\left[\begin{array}{ccc}
          2 & 1\\
          4 & 2\\
      \end{array}\right]
    \end{array}
  \end{equation*}
\end{minipage} &
\begin{minipage}{6cm}
  \begin{equation*}
    \begin{array}{ll}
      \text{bottom:} &
      \dfrac1{216}\left[\begin{array}{ccc}
          1 & 4 & 1\\
          4 & 16 & 4\\
          1 & 4 & 1
      \end{array}\right] \\[4mm]
      \text{center:} &
      \dfrac1{216}
      \left[\begin{array}{ccc}
          4 & 16 & 4\\
          16 & \underline{64} & 16\\
          4 & 16 & 4
      \end{array}\right] \\[4mm]
      \text{top:}& 
      \dfrac1{216}
      \left[\begin{array}{ccc}
          1 & 4 & 1\\
          4 & 16 & 4\\
          1 & 4 & 1 
      \end{array}\right]
    \end{array}  
  \end{equation*}
\end{minipage}
\end{tabular}

\section{Generalized Laplace operator}

%------------------------------------------------------------------------------------------------
\section{Propositions}
In this section some propositions are collected such that they can be referenced when needed.

\subsection{Divergence theorem}
\textit{Also called Gauss's theorem.}
Let $U \subset \R^d$ be a compact set with a piecewise smooth boundary $\p U$, $\bfF: U \to \R^d$ a continuously differentiable vector field. Then:
\begin{equation}\label{eq:gauss}
  \begin{array}{ll}
    \ds\int_U ∇\cdot\bfF(\bfx) \,\d \bfx = \ds\int_{\p U} \bfF(\bfx)\cdot \bfn\,\d \bfx.
  \end{array}
\end{equation}
For $d=2$ one gets \emph{Stoke's theorem}.

\subsubsection{Corollary}
Replacing $\bfF$ of \eqref{eq:gauss} by ${f\,\bfF}$ yields the following proposition:

For a differentable function $f: U \to \R$ and a vector field $\bfF: U \to \R^d$ it holds:
\begin{equation}\label{eq:gauss1}
  \begin{array}{ll}
     \ds\int_U f(∇\cdot\bfF) \,\d \bfx = \ds\int_{\p U} (f\,\bfF)\cdot\bfn\,\d \bfx -\ds\int_U \bfF \cdot ∇f \,\d \bfx
  \end{array}
\end{equation}
Now set $\bfF\equiv (1,0,\dots), (0,1,\dots), \dots$ to get the following vector-valued identity:

For a differentable function $f: U \to \R$ it holds:
\begin{equation}
  \begin{array}{ll}
    \ds\int_U ∇f(\bfx) \,\d \bfx = \ds\int_{\p U} f(\bfx)\,\bfn\,\d \bfx
  \end{array}
\end{equation}

\subsection{Classical Stoke's theorem}
Let $U\subset \R^3$ be an open set, $V$ a 2-manifold in $U$ with boundary $\p V$ and $\bfF: U \to \R^3$ a continuously differentiable vector field. Then:
\begin{equation}
  \begin{array}{ll}
    \ds\varointctrclockwise_{\p V} \bfF\cdot\d s = \ds\int_{V} \big(∇\times \bfF\big) \cdot \bfn \,\d\bfx,
  \end{array}
\end{equation}
where $\bfn$ is the normal on the surface $V$.

\subsection{Integration on manifolds}
In the following it is outlined how to integrate on 1D and 2D domains that are embedded in $\R^d$. The formalism of manifolds is omitted for simplicity.

\subsubsection{1D curve integrals}
Let $U\subset \R$ be an open set (the parameter space) and $\Phi:U \to \R^d$ a smooth mapping that defines a curve $\Omega=\Phi(U)$ embedded in $\R^d$. An integrable function $g:\Omega \to \R$ can then be integrated as follows:
%
\begin{equation}\label{eq:integration_transformation_1d}
  \begin{array}{ll}
    \ds\int_{\Phi(U)} g(\bfx) \,\d\bfx = \int_{U} g\big(\Phi(\xi)\big)\,\Vert \Phi'(\xi)\Vert_2 \, \d \xi
  \end{array}
\end{equation}


\subsubsection{2D surface integrals}
Let $U \subset \R^2$ be an open set (parameter space), $\Phi:U \to \Phi(U)=:\Omega \subset\R^3$ a diffeomorphism, $\Phi$ maps parameters $\bfxi=(\xi_1,\xi_2) \in U$ to points in world space $\bfx \in \Omega$. The inverse map $\Phi^{-1} : \Omega \subset \R^3 \to \R^2$ assigns coordinates $(\xi_1,\xi_2)$ to each point $\bfx\in\Omega$. We name $\Phi^{-1}(\bfx) = (x(\bfx),y(\bfx))$ in the following formula. The integration of a 2-dimensional function $g:\Omega \to \R$ is performed as follows.
%
\begin{equation}\label{eq:integration_transformation_2d}
  \begin{array}{ll}
    \ds\int_{\Phi(U)} g(\bfx) \,\d\bfx  
    &  = \ds\int_{U} g\big(\Phi(\bfxi)\big) 
    \sqrt{\det \big(D\Phi(\bfxi)^\top D\Phi(\bfxi)\big)}\,\d\bfxi\\[4mm]
    
    & = \ds\int_{U} g\big(\Phi(\bfxi)\big) 
    \sqrt{\det\mat{\d{\Phi}{\xi_1} \cdot \d{\Phi}{\xi_1} & \d{\Phi}{\xi_1} \cdot \d{\Phi}{\xi_2}  \\[4mm]
    \d{\Phi}{\xi_1} \cdot \d{\Phi}{\xi_2} & \d{\Phi}{\xi_2} \cdot \d{\Phi}{\xi_2}}}\,\d\bfxi\\[4mm]
    
     & = \ds\int_{U} g\big(\Phi(\bfxi)\big) 
    \sqrt{\Big\Vert \d{\Phi}{\xi_1}\Big\Vert_2^2\,\Big\Vert \d{\Phi}{\xi_2}\Big\Vert_2^2 - \Big(\d{\Phi}{\xi_1} \cdot \d{\Phi}{\xi_2}\Big)^2 }\,\d\bfxi\\[4mm]
    
    
  \end{array}
\end{equation}

\subsubsection{Substitution on domains with same dimensionality}
\emph{Integration by substitution}, \textit{German \say{Transformationssatz}}, also \emph{change of variables rule}.
Let $U \subset \R^d$ be an open set, $\Phi:U \to \Phi(U) \subset\R^d$ a diffeomorphism ($\Phi$ bijective and continuously differentiable, inverse map $\Phi^{-1}$ also continuously differentiable).

Then $g:\Phi(U) \to \R$ is integrable on $\Phi(U)$ if and only if the function $\bfxi \mapsto g(\Phi(\bfxi))\,|\det(D\Phi(\bfxi))|$ is integrable on $U$. It holds:
\begin{equation}\label{eq:integration_transformation_3d}
  \begin{array}{ll}
    \ds\int_{\Phi(U)} g(\bfx)\,\d \bfx = \ds\int_U g(\Phi(\bfxi))\,|\det(D\Phi(\bfxi))|\,\d \bfxi,
  \end{array}
\end{equation}
where $D\Phi$ is the Jacobian of $\Phi$.

\subsubsection{Summary}
The transformation rules \cref{eq:integration_transformation_1d,eq:integration_transformation_2d,eq:integration_transformation_3d} can be summarized in a unified form as follows.

Let $U \subset \R^d, d\in\{1,2,3\}$ be an open set (parameter space), $\Phi:U \to \Phi(U)=:\Omega \subset\R^d$ a diffeomorphism that maps parameters $\bfxi \in U$ to points in world space $\bfx \in \Omega$. A function defined in parameter space, $f:U\to \R$, can then be integrated as follows in world space.
%
\begin{equation}\label{eq:integration_transformation_dd}
  \begin{array}{lll}
    \ds\int_{\Phi(U)} f\big(\Phi^{-1}(\bfx)\big)\,\d \bfx = \ds\int_U f(\bfxi)\,J_d(\bfxi)\,\d \bfxi,
  \end{array}
\end{equation}
where the definition of $J_d(\bfxi)$  depends on the dimension $d$ as follows:
%
\begin{equation*}
  \begin{array}{rll}
    J_1(\xi) &= \Vert \Phi'(\xi) \Vert_2 &\quad \text{for }d=1, \bfxi=\xi \in U \subset \R\\[4mm]
    J_2(\bfxi) &= \sqrt{\det \big(D\Phi(\bfxi)^\top D\Phi(\bfxi)\big)} &\quad \text{for }d=2, \bfxi\in U \subset \R^2, \phi^{-1}(\bfx) =: \big(x(\bfx), y(\bfx)\big)\\[4mm]
    J_3(\bfxi) &= |\det (D\Phi\big(\bfxi)\big)| &\quad \text{for }d=3, \bfxi\in U \subset \R^3
  \end{array}
\end{equation*}

% -------------- Literaturseite --------------------
%\newpage
%\bibliography{literatur}{}
%\bibliographystyle{abbrv}

% -------------- Anhang ------------
%\appendix
%\input{8_anhang.tex}

\end{document}
