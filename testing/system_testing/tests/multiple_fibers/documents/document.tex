%===============================================================================
%===============================================================================
%
\clearpage
%
\section{Monodomain on Biceps Brachii geometry}
  %
  The name of these system tests is \lstinline{multiple_fibers}, however only one fiber is simulation for this test case.
  The monodomain equation \eqref{eq:monodomain1d} is used with the Hodgkin-Huxley model.
  
  The used parameters are
  \begin{equation*}
    \begin{array}{lll}
      c = \text{Conductivity/(Am*Cm)},\qquad \Omega = [0,50], \qquad t_\text{end}=50.
    \end{array}
  \end{equation*}
  The discretization parameters are
  \begin{table}[h!]
    \begin{center}
      \begin{tabular}{l|l}
        \textbf{Parameter} & \textbf{Value}\\
        \hline
        Number of elements & 1266\\
        splitting timestep $dt_\text{3D}$ & $10^{-1}$\\
        timestep of diffusion term $dt_\text{1D}$ & $10^{-5}$\\
        timestep of ODEs $dt_\text{0D}$ & $5\cdot 10^{-5}$
      \end{tabular}
    \end{center}
    \caption{discretization parameters}
    \label{tab:table_monodomain2}
  \end{table}

\subsection{Result summary}
%
\begin{figure}[h!]
  \animation{../tests/multiple_fibers/results/fibre_1_.mp4}
  \caption{\lstinline{fibre_1}}
\end{figure} 
%
\begin{figure}[t]%
  \centering%
  \includegraphics[width=0.9\textwidth,keepaspectratio]{../tests/multiple_fibers/results/fibre_1_.pdf}%
  \caption{\lstinline{fibre_1}}
\end{figure}%
%

\lstinputlisting[breaklines,basicstyle=\tiny]{../tests/multiple_fibers/results/log_recent_fibre_1_.txt}
%
%===============================================================================
%===============================================================================
