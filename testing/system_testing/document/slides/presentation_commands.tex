\DeclareGraphicsRule{*}{pdf}{*}{}

% this is really really discouraged as it overwrites whatever beamer style you define within your
% .sty file and causes endless debugging sessions :////
%\useinnertheme{rounded}%rectangles}

\newcommand{\ES}{ESPResSo}

\newcommand{\Pfeil}{\ensuremath{\Rightarrow}}
\newcommand{\Order}{\ensuremath{\mathcal{O}}}

\newcommand{\bbN}{\mathbb{N}}
\newcommand{\bbR}{\mathbb{R}}

\newcommand{\Var}{\operatorname{Var}}
\newcommand{\Vol}{\operatorname{Vol}}

\newcommand{\vx}{{\vec{x}}}
\newcommand{\vy}{{\vec{y}}}
\newcommand{\valpha}{{\vec{\alpha}}}
\newcommand{\vl}{{\vec{l}}}
\newcommand{\vi}{{\vec{i}}}
\newcommand{\vj}{{\vec{j}}}
\newcommand{\vh}{{\vec{h}}}
\newcommand{\vone}{{\vec{1}}}
\newcommand{\vzero}{{\vec{0}}}
\newcommand{\vli}{{\vl,\vi}}

\newcommand{\ddx}{\frac{d}{dx}}
\newcommand{\mix}{\textrm{mix}}
%\newcommand{\D}{\textrm{d}}

%\DeclareMathOperator*{\argmin}{arg\,min}
%\DeclareMathOperator*{\argmax}{arg\,max}

\newcolumntype{C}[1]{>{\centering\arraybackslash}p{#1}} % zentrierte Spalten mit Breitenangabe
\newcolumntype{R}[1]{>{\raggedleft\arraybackslash}p{#1}} % rechtsbuendig mit Breitenangabe
\newcolumntype{Z}{>{\centering\arraybackslash}X} % zentrierte Spalten mit automatischer Breite
% Zu Beginn der Zelle angeben:
\newcommand{\ctab}{\centering\arraybackslash } % Tabellenabschnitt zentrieren
\newcommand{\rtab}{\raggedleft\arraybackslash} % Tabellenabschnitt rechtsbuendig
\newcommand{\ltab}{\raggedright\arraybackslash} % Tabellenabschnitt linksbuendig

% draw arrays
\newcounter{listcount}
\newcounter{totcount}
\newcommand{\printarray}[2][1em]{% \printarray[<width>]{<array list>}
  \unskip \setcounter{totcount}{0}% Reset totcount counter
  \renewcommand*{\do}[1]{\stepcounter{totcount}}% Count elements
  \docsvlist{#2}% Process list a first time to obtain # of elements
  \setcounter{listcount}{0}% Reset listcount counter
  \renewcommand*{\do}[1]{%
    \stepcounter{listcount}% Move to next element
    \framebox[#1][c]{\rule{0pt}{1.5ex}\smash{\ensuremath{##1}}}%
    \ifnum\value{listcount}<\value{totcount}\thickspace\fi
  }
  \docsvlist{#2}% Process list a second time to typeset each element
}

\newcommand{\Hide}[1]{}\let\ignore\Hide

\lstdefinelanguage{cPython}[]{Python}{
  basicstyle=\ttfamily\small,
  stringstyle=\color{red},
  keywordstyle=\color{blue},
  showstringspaces=false,
%morekeywords={map, filter},
  commentstyle=\color{gray}\slshape,
  framexleftmargin=1mm, framextopmargin=1mm, frame=shadowbox,
  rulesepcolor=\color{tumblau}
}

\lstnewenvironment{python}[1][]{
  \lstset{
    language=cPython,#1
  }
}{}

\newcommand{\pyinline}[1]{\lstinline[language=cPython]@#1@}
